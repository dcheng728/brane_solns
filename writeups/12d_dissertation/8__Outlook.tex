\section{Concluding Remarks and Outlook}
% The question of whether the type IIB string theory has a 12d origin remains open, but several lines of evidence suggest at least a partial 12d structure.




In this paper we have explored the better-understood, and the still-speculative corners of the 12d interpretations on the type IIB theory.
From the brane perspective, the connections between D7 and KK-monopoles, and between D(-1) and pp-waves strongly suggests that the axio-dilaton action be associated with 12d gravity.
For the D3 brane, electromagnetic SL(2, Z) duality on the worldvolume is possible only when accompanied by the corresponding SL(2, Z) transformations of the bulk fields.
This connection suggests that the D3 may be a key object for understanding the origin of type IIB duality.
The structure of the effective actions provides a second, more speculative line of investigation.
In particular, the modular completions that render the SL(2, Z) duality exact, such as the appearance of non-holomorphic Eisenstein series in higher-derivative couplings, may have a 12d interpretation.
Throughout this review, we have also identified several directions forward, the most prominent one being a more general 12d effective action ansatz, and a systematic study of 5-point amplitudes in 10d.

The role of SL(2,Z) acquires further significance in the context of the AdS/CFT correspondence.
Recently, it was shown that an M2 brane wrapping a circle at the boundary of the $AdS_4\times S^7$ background reproduces, via a one-loop computation of its worldvolume effective action, the subleading $1/N$ corrections to Wilson loop observables in the dual ABJM theory \cite{Giombi:2023vzu, Tseytlin:2024euk}. 
This naturally raises the question of whether an analogous construction exists for the $AdS_5\times S^5$ background.
Addressing this question requires identifying the type IIB counterparts of the $AdS_4\times S^7$ geometry and the M2 brane.
A sharper understanding of the relationship between type IIB string theory, and its potential 12d interpretation may therefore shed light on the possibility of such a correspondence.

