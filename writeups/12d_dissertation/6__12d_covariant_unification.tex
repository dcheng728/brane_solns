\section{A Covariant Unification in 12d}
The various brane-related evidence for an effective 12d interpretation of the type IIB theory suggests a very specific 12d interpretation of the axio-dilaton sector.
In addition, one seeks a 12d interpretation of the type IIB RR and NSNS form fields.
In this section, we gather the various 12d insights obtained in the previous section from the type IIB branes, and present a 12d covariant unification of $SL(2,R)\times SO(9,1)$ form fields.
We will use calligraphic letters to denote fields in the higher dimension\footnote{
It is standard in the literature to use $\mathcal C$ to denote the Euclidean compact scalar.
We will adopt that $\mathcal C$ is the 10d compact scalar.
This should not cause any confusions because we will not work with any 12d scalars, as all type IIB scalars are uplifted into the 12d metric $\mathcal G_{MN}$.
% This should not cause confusion because we will only be working with higher forms rank-3 or above in 12d.
}.
The 10d metric $g_{mn}$ with $m,n=0,1,...,9$ will be interpreted as embedded inside a 12d metric $\mathcal G_{MN}$, with $M,N=0,1,...,11$. We will use the calligraphic $\mathcal R$ to denote the Ricci scalar computed with $\mathcal G_{MN}$, and use $\mathcal F_{n+1} = d\mathcal C_{n}$ to denote form fields in 12d.

Let the 12d coordinates be parameterized by $(x^m,u,v)$, and let $M_{ij}$ be the $2\times 2$ metric on the torus.
The key insight from the previous section is that one shall consider the 12d metric embedding given by
\begin{equation}
\mathcal{G}_{MN}=\begin{pmatrix}
g_{mn} & 0 \\
0 &M_{ij}
\end{pmatrix},\quad
M_{ij}
=\frac{1}{\tau_2}\begin{pmatrix}
1& \tau_1\\
\tau_1 & \tau_1^2+\tau_2^2
\end{pmatrix}.
\label{12d metric ansatz}
\end{equation}
The axio-dilaton action may be written as a 12d Einstein-Hilbert action compactified on $T_2$. In particular,
\begin{equation}
\begin{aligned}
\frac{1}{2\kappa_{10}^2}\int d^{10}x\sqrt{-g}\left(
R - \frac{\partial\tau \partial\bar\tau}{2\tau_2^2}
\right)
&=
\frac{Vol(T_2)}{2\kappa_{12}^2}\int d^{10}x \sqrt{-\mathcal{G}} \mathcal{R}\\
&=\frac{1}{2\kappa_{12}^2}\int_{T_2}dudv\int d^{10}x\sqrt{-\mathcal G}\mathcal{R},
\end{aligned}
\label{axio-dilaton 12d uplift}
\end{equation}
where we have schematically defined $\kappa_{12}$ by 
\begin{equation}
\frac{1}{\kappa_{10}^2} = \frac{Vol(T_2)}{\kappa_{12}^2}
=\frac{1}{\kappa_{12}^2}\int_{T_2}*_2 1,\quad
[\kappa_{12}^2] = L^{10}.
\end{equation}
The 3-form field strengths form an SL(2,R) doublet. 
To unify them in 12d we define a 12d 4-form field strength with exactly one leg on the torus:
\begin{equation}
\mathcal{F}_4=d\mathcal C_3 = H_3\wedge du + F_3\wedge dv,\quad
\mathcal C_3 \equiv B_2\wedge du + C_2\wedge dv.
\end{equation}
Using the 12d metric ansatz \eqref{12d metric ansatz}, the 10d 3-form field strengths and their axio-dilaton couplings follow from contracting $\mathcal F_4$:
\begin{equation}
\begin{aligned}
|\mathcal F_4|\bigg|_{\mathcal{G}_{MN}}
&=M^{uu}|H_3|^2+2M^{uv}|F_3\cdot H_3|+M^{vv}|F_3|^2\\
&=\left(
e^{-\Phi}|H_3|^2 + e^\Phi|F_3-CH_3|^2
\right)\bigg|_{g_{mn}}.
\end{aligned}
\end{equation}
Finding a 12d interpretation for the 5-form sector is trickier. 
One observes that the composite, SL(2, R) singlet 5-form 
\begin{equation}
\tilde F_5 = F_5 - \frac12 C_2 \wedge H_3 + \frac{1}{2}B_2\wedge F_3    
\end{equation}
can not be sensibly constructed in 12d, due to the lack of a pair of form field potential and strength with ranks that sum to 5 in 12d.
However, the 10d self-dual 5-form field strength admits two possible uplifts to 12d:
\begin{equation}
\mathcal{F}_5 = F_5,\quad \mathcal{F}_7=F_5\wedge du\wedge dv.
\end{equation}
Then note that
\begin{equation}
\begin{aligned}
|\mathcal{F}_5|^2\bigg|_{\mathcal{G}_{MN}} 
&= |F_5|^2\bigg|_{g_{mn}},\\
|\mathcal{F}_7|^2\bigg|_{\mathcal{G}_{MN}}
&=\det(M_{ij})|F_5|^2\bigg|_{g_{mn}} = |F_5|^2\bigg|_{g_{mn}},
\end{aligned}
\end{equation}
where in the last equality we used $\det (M_{ij})=1$.
Then we may define a 12d 7-form
\begin{equation}
\tilde{\mathcal F}_7 \equiv \mathcal F_7 + \frac{1}{2}\mathcal{C}_3 \wedge \mathcal F_4
\end{equation}
that exactly supplies the type IIB composite 5-form contribution upon contraction in 12d:
\begin{equation}
|\tilde{\mathcal{F}}_7|^2\bigg|_{\mathcal{G}_{MN}} = 
\det(M_{ij})|\tilde F_5|^2\bigg|_{g_{mn}}.
\end{equation}
The 10d self-duality condition on $F_5$ may then be written as a 12d Hodge duality
\begin{equation}
\mathcal F_7 = *_{12}\mathcal F_5\quad
\Leftrightarrow\quad
F_5 = *_{10}F_5.
\label{12d interpretation of 10d self-duality}
\end{equation}
The 10d Chern-Simons term can also be obtained using the 12d potentials and their corresponding field strengths:
\begin{equation}
\frac{1}{2}\int_{T_2\times \mathbb{R}^{1,9}} \mathcal C_4\wedge \mathcal{F}_4\wedge\mathcal{F}_4=
Vol(T_2)\int_{\mathbb{R}^{1,9}}C_4\wedge H_3\wedge F_3.
\end{equation}
We note that reproducing the type IIB Chern-Simons term in 12d necessitates the inclusion of both 4- and 5-form field strengths.
One can write down a ``12d"\footnote{
Not dynamical 12d, but dynamical 10d times a 2-torus.
}
action
\begin{equation}
\begin{aligned}
S_{IIB}=
S_{\text{``12"}}
= \frac{1}{2\kappa_{12}^2}\int_{T_2}dudv\int d^{10}x
\sqrt{-\mathcal G}
\left(
R - \frac{1}{2}|\mathcal F_4|^2-\frac{1}{4}|\tilde{\mathcal{F}}_7|^2
\right)-\frac{1}{4\kappa_{12}^2}\int_{T_2\times \mathbb{R}^{1,9}}\mathcal C_4\wedge \mathcal F_4\wedge \mathcal F_4.
\end{aligned}
\label{12d action}
\end{equation}
The $\mathcal F_7$ does not arise from an independent degree of freedom, it is related to $\mathcal C_4$ by $*_{12}d\mathcal C_4 = \mathcal F_7$.
The action \eqref{12d action} is exactly the Einstein frame type IIB action \eqref{Einstein frame IIB action} with $g_s=1$.
The 2d integrand is just a repackaging of $1/\kappa_{10}^2$, and it is likely that $\mathcal C_3, \mathcal C_4$ do not furnish independent degrees of freedom, as has been discussed in \cite{Tseytlin:1996ne}.