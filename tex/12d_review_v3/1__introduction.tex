\section{Introduction}
The SL(2, Z) symmetry, acting on the axio-dilaton and exchanging the fundamental string with the D1 brane, is essential to the non-perturbative consistency of the type IIB string theory in 10d. 
Its underlying origin, however, remains obscure. 
Although it has long been conjectured that this duality reflects an underlying 12d origin, the precise relation remains unclear.


Following the construction of 11d supergravity \cite{Nahm:1977tg,Cremmer1978}, the existence of SO(10, 2) Majorana-Weyl spinors and their corresponding superalgebra has motivated discussions of a 12d supersymmetric theory \cite{Castellani:1982ke,Blencowe:1988sk}.
Later, through the study of the 6d (2,0) gravity theory \cite{Hull:1995xh,Witten:1995em}, and the 7-brane backreacted vacua \cite{Vafa:1996xn}, more evidence for an effective 12d interpretation of the type IIB string theory have emerged.
The latter \cite{Vafa:1996xn} has developed into a field of study on string compactifications and particle phenomenology called ``F-theory".
Meanwhile, the worldvolume and bulk SL(2, Z) duality of the D3 brane had been discussed in the context of 12d \cite{Tseytlin:1996it}, and the D(-1) became understood as a 12d pp-wave reduced on a non-compact torus \cite{Tseytlin:1996ne}.

However, the construction of a 10+2d supergravity \cite{Nishino:1997gq} has faced persistent conceptual obstructions, due to its non-compact little group and the lack of a momentum generator in its algebra \cite{Hewson:1997wv}.
To this day, no consistent 10+2d supergravity with 32 supercharges that reduces to the type IIB supergravity has been constructed. 
On the other hand, effective 12d perspectives, arising from branes \cite{Tseytlin:1996ne}, dualities \cite{Tseytlin:1996it}, and effective actions \cite{Minasian:2015bxa} appear to be more consistent, and have provided more insights into the type IIB theory and its SL(2, Z) duality.
The aim of this review is to organize what has been attempted, what is understood, and what might be the way forward.


This paper is organized as follows. We first introduce the type IIB theory and discuss its puzzles.
For completeness, we review various attempts at formulating a 10+2d supergravity theory.
Then we move to our main interest, that is the 12d perspectives arising from branes and effective actions.
We begin with the sector that most robustly admits a 12d interpretation: the axio-dilaton sector.
We show that the D7 background may be viewed as the 12d Kaluza-Klein monopole reduced on a torus, and that the D(-1) background can be obtained by compactifying the 12d pp-wave.
In parallel, we present the analogous stories on the type IIA side, and discuss the consistencies related to supersymmetry.

We next examine the D3 brane worldvolume action, noting that consistency requires the identification between the bulk SL(2, Z) S-duality of type IIB supergravity and the SL(2, Z) electromagnetic duality acting on the D3 brane worldvolume fields.
This interplay suggests that the D3 brane may acquire a special role in understanding 12d insights into the type IIB S-duality. 

In section 6, we present a framework that embeds the full two-derivative type IIB action into a 12d-covariant formulation, in which the 10d self-duality condition on the 5-form is implemented via a 12d Hodge duality.
Then we discuss how the higher-derivative corrections to the type IIB action may be interpreted from a 12d perspective.
Lastly, we remark that the Kaluza-Klein modes in the type IIA and type IIB supergravity can serve as surrogates for the D0 and D(-1) backgrounds in computations of the string effective action.
This viewpoint may offer insights into a possible 12d interpretation of these effects, and we outline how this connection might be further substantiated.