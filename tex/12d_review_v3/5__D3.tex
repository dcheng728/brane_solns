\section{The D3 Brane and Self-Duality}
The low-energy dynamics of a D3 brane in a given type IIB supergravity background are described by the Dirac-Born-Infeld action supplemented by a Wess-Zumino term, with background metric, NSNS and RR fields entering through the appropriate pullbacks to the brane worldvolume \cite{Leigh:1989jq,Polchinski:1998rr}.
In particular, the bosonic sector of the D3 low-energy action takes the form \cite{Tseytlin:1996it}
\begin{equation}
\begin{aligned}
S_{D3}
=\int d^4x
\Bigg[
&\sqrt{-\det (\hat{g}_{mn} + e^{-\hat \Phi/2} \mathfrak{F}_{mn})}\\
&+\frac{1}{8}\epsilon^{mnkl}
\Bigg(
\frac{1}{3}\hat C_{mnkl}
+2\hat C_{mn}\mathfrak{F}_{kl}
+C\mathfrak{F}_{mn}\mathfrak{F}_{kl}
\Bigg)
\Bigg]
+\text{higher order},\\
&\mathfrak{F}_{mn} \equiv\partial_m A_n-\partial_n A_m+\hat B_{mn}.
\end{aligned}
\label{D3 action}
\end{equation}
Here hats denote bulk fields pulled back onto the brane worldvolume, and $m,n$ label worldvolume indices.
The vector $A_m$ is the dynamical gauge potential on the brane worldvolume.
To perform worldvolume electromagnetic duality transformation, one introduces a Lagrange multiplier
\begin{equation}
\Lambda^{mn} = \epsilon^{mnkl}\partial_k \tilde A_l,
\end{equation}
where $\tilde A_l$ is the dual vector potential, with field strength $\tilde F_{pq}$.
Then after $F_{mn}$ is eliminated with the field equations of $\Lambda^{mn}$, one is left with the dual field $\tilde{F}_{mn}$.
It was shown \cite{Tseytlin:1996it} that as one performs the electromagnetic duality transformation, the D3 action \eqref{D3 action} is invariant only if one simultaneously performs the following SL(2, Z) bulk transformation:
\begin{equation}
e^{-\Phi}\to \frac{1}{e^{-\Phi}+e^\Phi C^2},\quad
C\to -\frac{C e^\Phi}{e^{-\Phi}+e^\Phi C^2},\quad
B_{\mu\nu} \to C_{\mu\nu},\quad
C_{\mu\nu}\to -B_{\mu\nu}.
\end{equation}
Since the action is also invariant under the axion shift $C \to C+1$, the symmetry is the full SL(2, Z). Each worldvolume SL(2, Z) transformation thus maps directly to a bulk type IIB duality transformation.
The bulk generator $T: \tau\to \tau+1$ corresponds to $\theta\to \theta + 2\pi$ on the D3 worldvolume, for the complexified coupling $\tau_{YM}\equiv \frac{\theta}{2\pi} + \frac{4\pi i}{g^2}$.
Meanwhile, the bulk generator $S:\tau\to-1/\tau$ is mapped to the electromagnetic duality transformation on the worldvolume field strengths.
Thus the SL(2, Z) duality of $\mathcal N=4$ SYM on the D3 worldvolume is intimately linked to the type IIB SL(2, Z) duality, which acts on the bulk NSNS and RR fields. 

For comparison, we recall a similar story on the type IIA side.
The analogy is not direct because there is no analogue of the SL(2, Z) duality in the type IIA theory.
The D2 action can be written as \cite{Schmidhuber:1996fy,Tseytlin:1996it}
\begin{equation}
S_{D2}
=\int d^3x \sqrt{-e^{-2\hat{\Phi}}\det (\hat{g}_{mn}+\mathfrak{F}_{mn})}
+\frac{1}{6}\epsilon^{mnl}[\hat C_{mnl}-3\hat C_m\mathfrak{F}_{nl}],\quad
\mathfrak{F}_{mn}\equiv 2\partial_{[m}A_{n]}-\hat B_{mn}.
\end{equation}
By performing the worldvolume vector-scalar duality transformation, we exchange the worldvolume vector $A_n$ for a scalar $\partial_n y$, and the action becomes
\begin{equation}
S_{D2}
=\int d^3 x\sqrt{-\hat{\mathcal{G}}} + \frac{1}{6}\epsilon^{mnl}\hat{\mathcal C}_{mnl},
\end{equation}
with
\begin{equation}
\hat{\mathcal G}_{mn} \equiv e^{-2\hat\Phi/3}\hat g_{mn}  + e^{4\hat\Phi/3}(\hat C_m-\partial_my)(\hat C_n-\partial_ny),\quad
\hat{\mathcal C}_{mnl}\equiv \hat{C}_{mnl} + 3 \hat{B}_{mn}\partial_l y.
\end{equation}
We see that if one interprets the worldvolume scalar $y$ as a 10d scalar pulled back onto the D2 worldvolume, then this is precisely the M2 action directly reduced on $S_1$ \cite{Bergshoeff:1987cm,Tseytlin:1996it,Schmidhuber:1996fy}.

Given the fundamental role of the M2 brane in 11d supergravity, this analogy suggests that some 3-brane might play an analogous role in a speculative 12d effective description of the type IIB theory.
A simple degrees of freedom count suggests that the D3 has enough fields to be embedded in 12d.
However, its two on-shell bosonic degrees of freedom arise from a gauge field, which makes a direct interpretation in terms of embedding coordinates difficult.
One can perform a double dimensional reduction of the D3 so that the vector decomposes into two scalars \cite{Jatkar:1996np,Kar:1997cx}, but this simply reproduces the standard relation between M-theory on $T_2$ and 9d supergravity \cite{Green:1997as}.


