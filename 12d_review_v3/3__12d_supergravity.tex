\section{A 10+2d Theory?}
There exists an algebra, which admits a 10+2d interpretation, that may unify the algebras of the various theories with 32 supercharges. 
Historically, this motivated a series of investigations at formulating a supersymmetric 10+2d theory of gravity.
These constructions turn out not to be directly relevant to our main focus, but we review them for completeness.
The objective here is to present these attempts in an organized fashion, discuss their obstructions and insufficiencies, and set the stage for the brane and effective action discussions later.

In 12d the complex Dirac spinor has 64 components and decomposes into two 32-component Weyl spinors.
In general, they are complex, but for the Lorentz group SO(10, 2), because
\begin{equation}
s-t=0\mod 8,
\end{equation}
one can impose a Majorana condition compatible with chirality, to find Majorana-Weyl spinors with 32 real components \cite{VanProeyen:1999ni}.
The corresponding superalgebra contains a 2-form and a self-dual 6-form charge
\cite{vanHolten:1982mx}
\begin{equation}
\text{Sym}^2[000001]_{SO(10, 2)} = [010000]_{SO(10, 2)} + [000020]_{SO(10, 2)}.
\label{OSp algebra}
\end{equation}
This algebra is often referred to as the $\mathrm{OSp}(1|32)$ algebra \cite{Bergshoeff:2000qu} or the F-theory algebra \cite{Hewson:1999tf}.
It reduces to the 11d algebra
\begin{equation}
\text{Sym}^2[00001]_{SO(10,1)} = [10000]_{SO(10,1)} + [01000]_{SO(10,1)} + [00002]_{SO(10,1)}.
\end{equation}
Historically, \eqref{OSp algebra} led to interest in formulating supergravity theories with signature 10+2d \cite{Castellani:1982ke}.
It was also understood that 2+2d branes are allowed to propagate in (10,2) spacetime \cite{Blencowe:1988sk}.
Interestingly, \eqref{OSp algebra} also reduces to the 10d type IIA and 10d IIB algebra \cite{Bergshoeff:2000qu,Bergshoeff:2000vg,Bergshoeff:2000vh}.
The BPS states in the OSp(1$|$32) algebra have been studied \cite{Ueno:1999xa}, and the various consistent fractions of preserved supersymmetry had been worked out \cite{Manvelyan:1998kx}.
At an algebraic level, ideas for the unification of the various dualities in 12, 13 dimensions \cite{Bars:1996dz,Bars:1996us,Bars:1996ab,Bars:1996hr}, and even 14 dimensions have been suggested \cite{Rudychev:1997ui,Bars:1997ug}.

However, the possibility of a 10+2d theory hinges on there being two time-like directions, and there are fundamental issues with formulating a supersymmetric theory with two times.
The little group of an SO(10, 2) theory is SO(9, 1), whose finite-dimensional irreducible representations are either unitary and trivial or non-unitary \cite{Weinberg:1995mt}.
For unitary representations of the supersymmetry algebra, we may write the supercharge anticommutator 
\begin{equation}
\{Q,Q^\dagger\}=2|Q|^2=\sum_n\Gamma_n Z_n.
\end{equation}
Because the RHS is symmetric and positive definite, we can diagonalize it.
Then we obtain fermionic raising and lowering operators, which implies that there are 256 states in the system. 
However, this can not be carried out if the representation is non-unitary.
One might say, in any known one-time supergravity theory with 32 supercharges, there are 128+128 states, so start with this many as well. 
But the gravitino representation of SO(10) already has 144 states, exceeding the budget for fermions. There indeed is a way to add up to 144 + 144 states for SO(10) without too many scalars: a single gravitino in the fermion sector plus a graviton and two 2-forms in the bosonic sector. But this does not seem related to any known 10d or 11d multiplets.


Another issue that accompanies a non-compact little group is negative-norm states, or ``ghosts".
Partially motivated by studying the 10+2d theory, a practical two-time framework has been developed \cite{Bars:1998cs,Bars:1998ui,Bars:1999nk,Bars:1999uq,Bars:2000qm}. 
The key ingredient is a local Sp(2, R) gauge symmetry acting on the phase-space variables, treated as a two-component vector.
Different gauge choices lead to different one-time systems that share a common (d-2, 2) parent description.
The various approaches to formulating 10+2d theories largely follow this logic.

There have been some attempts at 10+2d SYM \cite{Nishino:1996wp,Sezgin:1997gr,Nishino:1997hk,Nishino:1997ia,Nishino:1999ck} and 10+2d supergravity \cite{Nishino:1997gq,Nishino:1997sw,Nishino:1998qn}, but none of them has achieved satisfactory results, as they either need to introduce null projectors that explicitly break SO(10, 2),
or fail to construct vielbeins due to the lack of a momentum generator \cite{Hewson:1999tf} in the algebra \eqref{OSp algebra}. 

After it had been shown that the (2+2)-brane can propagate in 10+2d spacetime \cite{Blencowe:1988sk}, the $\mathcal N=(2,1)$ string had been studied \cite{Kutasov:1996fp, Kutasov:1996zm, Martinec:1996wn}. The $\mathcal N=1$ sector contains an effective 10+2d target space, with the $\mathcal N=2$ sector on a 2+2d target. 
Later $\mathcal N=1$ superstring with 2+2d target space was constructed \cite{Khviengia:1995pm,Lu:1995pn}.
Further work investigated whether self-dual gravity in 2+2 dimensions admits a stringy description \cite{Sezgin:1995fj} and related these self-dual 2+2d strings to supersymmetric membrane action with OSp(1$|$32) and OSp(8$|$2) subgroup structure \cite{Ketov:1996gr}.
In parallel, a Green-Schwarz type super 2+2d brane embedded in 10+2d background framework had been constructed \cite{Hewson:1996yh,Hewson:1997wv,Hewson:1998sw}.

There had also been investigations on higher dimensional bosonic field theories, that upon compactifications and consistent truncations, may produce the known theories.
In \cite{Khviengia:1997rh}, a 12d action with imaginary dilaton couplings had been suggested as
\begin{equation}
2\kappa_{12}^2S
=\int d^{12}x\sqrt{-\mathcal G}
\left[
\mathcal R-\frac{1}{2}(\partial\Phi)^2-e^{\frac{2i}{\sqrt{5}} \Phi}
\frac{1}{2}|\mathcal{F}_5|^2
-\frac{1}{2}e^{\frac{i}{\sqrt{5}}\Phi }|\mathcal{F}_4|^2
\right]
+\frac{\sqrt{3}}{4}\int \mathcal{C}_4\wedge d\mathcal{C}_3 \wedge d\mathcal{C}_3
\label{Khviengia action}
\end{equation}
where $[\kappa_{12}^2]=L^{10}$ and $\Phi$ here is a 12d dilatonic scalar.
We will use calligraphic letters $\mathcal{G}_{MN},\mathcal C_n,\mathcal{F}_{n+1}$ for higher dimensional fields and standard letters $g_{mn},C_n,F_{n+1}$ for lower dimensional fields. 
This will be discussed more clearly in a later section.
This action \eqref{Khviengia action} had been subsequently studied by \cite{Bakhtiarizadeh:2018upg} at higher derivatives. 
The most prominent feature of the proposal \eqref{Khviengia action} by \cite{Khviengia:1997rh} is the 12d dilaton and its imaginary couplings.
It had been introduced based on certain scalar invariant \cite{Lu:1995cs} across 11d and 10d type IIB brane solutions \cite{Khviengia:1997rh}.
% Their argument for the necessity of the dilaton is based on certain scalar invariant  in certain brane solutions and dimensional reduction ansatz.
% We will later propose an alternative 12d to 10d ansatz that is consistent with multiple lines of brane evidence.
% As we will see, a dilatonic scalar in 12d

More recently, it has been claimed that F-theory \cite{Vafa:1996xn} admits a 12d action \cite{Choi:2014vya,Choi:2015gia}, taking the form
\begin{equation}
2\kappa_{12}^2S = \int d^{11}x dy \sqrt{-\mathcal G}
\left(
\mathcal R-\frac{1}{2}|\mathcal F_5|^2
\right)
+\frac{1}{6}
\int \mathcal C_4\wedge \mathcal{F}_4 \wedge \mathcal{F}_4.
\label{Choi action}
% \tag{1504.00602 eq 2.3}
\end{equation}
Upon closer examination, we find \eqref{Choi action} is just 11d supergravity integrated over a spectator dimension\footnote{
We provide support for this claim in Appendix \ref{sec:Action for 11d on a circle}.
}.
In an attempt to unify the various string theory dualities, an SL(2, R)$\times R^+$ Exceptional Field Theory had been proposed \cite{Berman:2015rcc,Rudolph:2017vzy}.
The idea is to realize unified dualities with extended coordinates, but this is not a 12d theory.

Related insights on a 12d viewpoint include proposals in which the type IIB theory is an edge mode of a gapped 12d bulk theory, motivated by a 6d generalisation of the fractional quantum Hall effect \cite{Heckman:2017uxe}, as well as the perspectives that 2+2 signature spacetime may be a tool for studying Lorentzian QFT \cite{Heckman:2018mxl,Heckman:2022peq}. 
We also note a recent supergravity construction of a locally supersymmetric $SO(10,2)$-invariant action in $D=10+2$,
of MacDowell-Mansouri type and containing the Einstein-Hilbert term \cite{Castellani:2017vbi}, whose physical degrees of freedom and the relation to standard 10d/11d supergravity remains to be clarified.
