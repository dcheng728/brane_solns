\section{Effective Actions and their 12d Consistencies}
The type IIB supergravity \eqref{Einstein frame IIB action} is understood as the low energy effective field theory of the type IIB string theory.
Under higher-derivative and stringy corrections, the SL(2, R) symmetry is believed to be broken to the discrete group SL(2, Z) \cite{Schwarz:1995dk}.
If the type IIB theory admits certain 12d interpretation, such interpretation, and its implications on effective actions, must be consistent with the type IIB effective action and SL(2, Z).

In this section, we review the current understandings in SL(2, Z) as arising from higher-derivative corrections, and their 12d interpretations.
We will begin by reviewing how one obtains higher-derivative corrections in supergravity from perturbative string theory, most importantly how the SL(2, Z) invariance arises non-perturbatively from the D(-1) backgrounds.
Then we discuss current understandings on 12d interpretations of SL(2, Z), and identify certain directions in advancing them. 
We conclude by discussing a potential parallel, between type IIA and type IIB, where the D0 and D(-1) backgrounds are effectively accounted for by loops of higher-dimensional KK modes.

\subsection{Recap: Perturbative String Theory and Higher-Derivative Supergravity}
In principle, one can obtain the higher derivative supergravity effective actions through Feynman diagrams.
In practice, they are obtained from string amplitudes. 
We briefly recap how this is done.

There are two perturbative parameters in string theory: $\alpha',g_s$.
\begin{itemize}
    \item $\alpha'=l_s^2$ controls low-energy expansion. Small $\alpha'$ is the particle limit of string theory, where we enter field theory (supergravity) whose higher derivatives appear accompanied by $\alpha'$.
    \item $g_s$ (string coupling) counts the genus of the string worldsheet in string path-integrals. Any given string amplitude is a summation over worldsheet path-integrals of all genus
    \begin{equation}
        \mathcal A_n(\alpha') = \sum_{g=0}^\infty g_s^{2g-2+n} \mathcal A_n^{(g)}(\alpha').
    \end{equation}
\end{itemize}
The tree-level 4-point function in the type IIA and type IIB string theories is the Virasoro-Shapiro amplitude \cite{Green:2012oqa}, for the symmetric traceless massless modes in the NSNS sector of the string, they take the following form \cite{Green:1999pv}
\begin{equation}
\mathcal A_4
=-t_8^{\mu_1...\mu_8}t_8^{\nu_1...\nu_8}
\prod_{r=1}^4\zeta^{(r)}_{\mu_{2r}\nu_{2r}}k^{(r)}_{\mu_{2r-1}}k^{(r)}_{\nu_{2r-1}}
\times \frac{\alpha^{\prime 4}}{g_s^2}
\times\frac{64}{\alpha^{\prime 3}stu}
\frac{\Gamma[1-\tfrac{\alpha'}{4}s]\Gamma[1-\tfrac{\alpha'}{4}t]\Gamma[1-\tfrac{\alpha'}{4}u]}{\Gamma[1+\tfrac{\alpha'}{4}s]\Gamma[1+\tfrac{\alpha'}{4}t]\Gamma[1+\tfrac{\alpha'}{4}u]}
\end{equation}
where $s,t,u$ are Mandelstam variables, $\zeta^{(i)}_{\mu\nu}$ is the polarization of the symmetric traceless modes on a closed string, they may also be interpreted as polarizations of the spacetime graviton $h^{(i)}_{\mu\nu}$ in $g_{\mu\nu}=\eta_{\mu\nu} + h_{\mu\nu}$. 
The symmetry of $\zeta_{\mu\nu}$ implies $t_8^{\mu_1...\mu_8}t_8^{\nu_1...\nu_8}$ is symmetric under $\mu_i\leftrightarrow\nu_i$.
Its explicit form can be found in \cite{Schwarz:1982jn,Liu:2025uqu}.
We can expand the $t_8t_8$ contraction explicitly and put back the momenta as derivatives
\begin{equation}
\begin{aligned}
&t^{\mu_1...\mu_8}t^{\nu_1...\nu_8}
[\zeta^{(1)}_{\mu_2\nu_2}k^{(1)}_{\mu_1}k^{(1)}_{\nu_1}]
[\zeta^{(2)}_{\mu_4\nu_4}k^{(2)}_{\mu_3}k^{(2)}_{\nu_3}]
[\zeta^{(3)}_{\mu_6\nu_6}k^{(3)}_{\mu_5}k^{(3)}_{\nu_5}]
[\zeta^{(4)}_{\mu_8\nu_8}k^{(4)}_{\mu_7}k^{(4)}_{\nu_7}]\\
&=t^{\mu_1...\mu_8}t^{\nu_1...\nu_8}
[h^{(1)}_{\mu_2\nu_2,\mu_1\nu_1}]
[h^{(2)}_{\mu_4\nu_4,\mu_3\nu_3}]
[h^{(3)}_{\mu_6\nu_6,\mu_5\nu_5}]
[h^{(4)}_{\mu_8\nu_8,\mu_7\nu_7}].
\end{aligned}
\end{equation}
Using $g_{\mu\nu} = \eta_{\mu\nu}+h_{\mu\nu}$, one finds
\begin{equation}
\begin{aligned}
R_{\mu\alpha\nu\beta}
&=\frac{1}{2}[-h_{\alpha\beta,\mu\nu}+h_{\alpha\nu,\mu\beta}]
-\frac{1}{2}[-h_{\mu\beta,\alpha\nu}+h_{\mu\nu,\alpha\beta}]
+ O(h^2)\\
&=-2h_{\big[\alpha[\beta,\mu \big]\nu]}  + O(h^2).
\end{aligned}
\end{equation}
Putting back the $O(h^2)$ in the Riemann tensor, we find\footnote{
It is conventional to define the following contraction
\begin{equation}
t_8t_8 R^4\equiv
t_8^{\mu_1...\mu_8}t_8^{\nu_1...\nu_8}
R_{\mu_2\nu_2\mu_1\nu_1}
R_{\mu_4\nu_4\mu_3\nu_3}
R_{\mu_6\nu_6\mu_5\nu_5}
R_{\mu_8\nu_8\mu_7\nu_7}.
\end{equation}
}
\begin{equation}
t_8t_8h^4=t_8t_8R^4 + O(h^5).
\end{equation}
Now we can write the string amplitude with spacetime fields, which amounts to the following effective Lagrangian
\begin{equation}
\mathcal L_4\supset
-t_8t_8 R^4\frac{4}{\alpha^{\prime 3}stu}
\frac{\Gamma[1-\tfrac{\alpha'}{4}s]\Gamma[1-\tfrac{\alpha'}{4}t]\Gamma[1-\tfrac{\alpha\prime}{4}u]}{\Gamma[1+\tfrac{\alpha'}{4}s]\Gamma[1+\tfrac{\alpha'}{4}t]\Gamma[1+\tfrac{\alpha'}{4}u]}
+O(h^5).
\end{equation}
The $t_8t_8R^4$ has no dependence on $\alpha'$, so by expanding the fraction of Gamma functions in $\alpha'$ we obtain the low energy effective action of the type II string theory.
Going into the Einstein frame\footnote{
One goes into the Einstein frame by replacing $g^{(S)}_{\mu\nu}=g_s^{1/2}g^{(E)}_{\mu\nu}$ with other terms kept intact. 
One then also needs to account for the $\sqrt{-g^{(S)}}$ multiplying the Lagrangian, as well as the inverse metric for contractions on the mandelstam variables.
}
we find\footnote{
We had also used
\begin{equation}
\ln\Gamma(1-x)-\ln\Gamma(1+x) = 2\sum_{m\ge0}\frac{\zeta(2m+3)}{2m+3}x^{2m+3},
\end{equation}
\begin{equation}
\frac{s^3+t^3+u^3}{stu}=3,\quad
\frac{s^5+t^5+u^5}{stu}=\frac{5}{2}(s^2+t^2+u^2),\quad
\frac{s^7+t^7+u^7}{stu}=\frac{7}{4}(s^2+t^2+u^2)^2.
\end{equation}
} \cite{Green:1999pv}
\begin{equation}
\begin{aligned}
\mathcal{A}^{(E)}_4
=-4\pi^7(t_8t_8R^4)(\alpha')^4
\Bigg[&\frac{4^3}{\alpha^{\prime 3}stu}+2\zeta(3)\tau_2^{3/2}+
\zeta(5)\tau_2^{5/2}\frac{\alpha^{\prime 2}(s^2+t^2+u^2)}{4^2}\\
&+\frac{2}{3}\zeta(3)^2\tau_2^3\frac{\alpha^{\prime 3}(s^3+t^3+u^3)}{4^3}
\Bigg] +O(\alpha^8).
\end{aligned}
\end{equation}
This is the genus-zero 4-graviton effective action, common between type IIA and type IIB, expanded in $\alpha'$.
The $\alpha'$ parameter appears with zeta functions, while the kinematics has, at leading order, a nonlocal pole
\begin{equation}
\frac{t_8t_8 R^4}{stu}
\end{equation}
trailed by local operators
\begin{equation}
t_8t_8 R^4,\quad
t_8t_8 R^4(s^2+t^2+u^2),\quad
t_8t_8 R^4(s^3+t^3+u^3),\quad
t_8t_8 R^4(s^2+t^2+u^2)^2.
\end{equation}
The poles of the Virasoro-Shapiro amplitude occur at $s,t,u=0$. These correspond to massless exchanges and reproduce exactly the pole structure expected from the $s,t,u$-channel diagrams generated by the cubic graviton vertices.
The nonlocal $1/stu$ contribution is therefore attributed to the tree-level supergravity dynamics.
% \begin{figure}
%     \centering
%     \includegraphics[width=0.75\linewidth]{image.png}
%     \caption{Schematic illustration of how $\frac{1}{stu}$ is attributed to tree-level amplitudes.}
%     \label{fig:stu}
% \end{figure}
By contrast, the trailing contributions are local in the low-energy expansion.
They are attributed to higher derivative corrections, usually accounted for by introducing terms denoted $D^{2k}R^4$, defined appropriately to absorb $(s^a+t^a+u^a)^b$.
The $t_8t_8R^4$ kinematics is often accompanied by $\epsilon_{10}\epsilon_{10} R^4$, however the $\epsilon_{10}\epsilon_{10}$ contributions vanish at 4-point and begins to contribute at 5-point amplitudes.

% It is also standard to define the combination
% \begin{equation}
% R^4 = t_8t_8 R^4 + \epsilon_{10}\epsilon_{10}R^4
% \end{equation}
% as the $\epsilon_{10}\epsilon_{10}$ contraction accompanies the $t_8 t_8$

The next order correction comes from genus-1 amplitudes \cite{Green:1999pv}.
It is also common in both type IIA and IIB theories, and comes with the $t_8 t_8 R^4$ factor. 
Combining the genus-0 and genus-1 amplitudes, we have the local effective actions
\begin{equation}
\mathcal{A}_4 = -4\pi^7 (t_8 t_8 R^4)(\alpha')^4
\left[
2 \zeta(3)\tau_2^{3/2}
+\frac{2\pi^2}{3}\tau_2^{-1/2}
\right]
+O(\alpha^{\prime 5}).
\end{equation}
Non-renormalization theorems suggest the perturbative corrections to $R^4$ terminate here at one-loop \cite{Green:1997tv,Kehagias:1997cq}.
Note that $\mathcal{A}_4$ as given above no longer has SL(2, R) symmetry. 
In fact, the SL(2, R) symmetry of the type IIB supergravity is only present at the two-derivative level together with corrections that do not depend on $\tau$.
As soon as higher-derivative terms with non-trivial dependence on $\tau$ enter, SL(2, R) is explicitly broken, with the discretized SL(2, Z) restored when contributions from terms that are non-perturbative in $g_s$ are included. 

In particular, the type IIB string path integral requires summing over the D(-1) backgrounds \eqref{D-1 background}.
The single and multi-charged D(-1) backgrounds give rise to non-perturbative $R^4$ corrections \cite{Green:1997tv,Kehagias:1997cq} of the form
\begin{equation}
\mathcal A_{D(-1)}\propto (\alpha')^4t_8t_8R^4\sum_{m,n\ge1}\left(\frac{m}{n^3}\right)^{1/2}
(e^{2\pi i mn\tau}+e^{-2\pi i mn\bar{\tau}})
\left(
1+\sum_{k=1}^\infty(4\pi mn\tau_2)^{-k}\frac{\Gamma[k-1/2]}{\Gamma[-k-1/2]k!}
\right).
\end{equation}
When combined with the genus-0 and genus-1 perturbative contributions, these D(-1) terms assemble into the modular-invariant, non-holomorphic Eisenstein series
\begin{equation}
\begin{aligned}
E_{3/2}(\tau,\bar\tau)
&=\sum_{(m,n)\neq(0,0)}\frac{\tau_2^{3/2}}{|m+n\tau|^3}\\
&=2\zeta(3)\tau_2^{3/2}+\frac{2\pi^2}{3}\tau_2^{-1/2}\\
&\quad+4\pi^{3/2}\sum_{m,n\ge1}\left(\frac{m}{n^3}\right)^{1/2}
(e^{2\pi i mn\tau}+e^{-2\pi i mn\bar{\tau}})
\left(
1+\sum_{k=1}^\infty(4\pi mn\tau_2)^{-k}\frac{\Gamma[k-1/2]}{\Gamma[-k-1/2]k!}
\right),
\end{aligned}
\end{equation}
thereby restoring the SL(2, Z) duality symmetry.

\subsection{12d Effective Corrections}
From our previous discussion, the type IIB 4-graviton effective action takes on the following schematic form \cite{Kehagias:1997cq,Green:1997tv}
\begin{equation}
L^{(3)} \propto E_{3/2}(\tau,\bar\tau)(t_8t_8 + \epsilon_{10}\epsilon_{10})R^4
\label{10d IIB 4-graviton effective action}
\end{equation}
The rest of the 4-point effective action consists of axio-dilaton and is SL(2, Z) invariant, they can be found in \cite{Policastro:2006vt,Policastro:2008hg}.
To produce the 10d effective action \eqref{10d IIB 4-graviton effective action},
a ``12d effective action" had been proposed as \cite{Minasian:2015bxa} 
\begin{equation}
\mathfrak{L}^{(3)} \propto E_{3/2}(\tau,\bar\tau)(\mathfrak{t}_8\mathfrak{t}_8 +{\varepsilon}_{12}{\varepsilon}_{12})\mathcal{R}^4,
\label{Liu 12d effective action}
\end{equation}
where $\mathfrak{t}_8$ is a 12d uplift of $t_8$\footnote{
The definition of $\mathfrak{t_8}$ can be found in \cite{Minasian:2015bxa}, it involves contractions of the 12d metric \eqref{12d metric ansatz}.},
and $\varepsilon_{12}$ is the 12d Levi-Civita tensor.
It was shown that \eqref{Liu 12d effective action} reduced on the 12d metric ansatz \eqref{12d metric ansatz} produces, at 4-point, the effective action in the axio-dilaton sector \cite{Policastro:2006vt,Policastro:2008hg}.
However, it was later shown that \eqref{Liu 12d effective action} is inconsistent with 10d type IIB amplitudes at 5-point \cite{Liu:2025uqu}.

We now comment on certain limitations of the proposed 12d effective action \eqref{Liu 12d effective action} and outline possible directions forward.
First, $\epsilon_{10}\epsilon_{10}R^4$ vanishes at 4-point, so the result of \cite{Minasian:2015bxa} on $\epsilon_{10}\epsilon_{10}R^4$ was that $\varepsilon_{12}\varepsilon_{12}\mathcal R^4$ vanishes at 4-point as well, after reducing $\mathcal G_{MN}$ to $g_{mn},\phi,C$.
The correspondence would be significantly stronger if non-vanishing components of $\epsilon_{10}\epsilon_{10}R^4$ could be verified, e.g. for 5-point amplitudes.
Unfortunately, this does not occur \cite{Liu:2025uqu}.

A second issue concerns the interpretation of the $t_8t_8 R^4$ term.
The $t_8$ tensor is originally defined by traces over the gamma matrices of SO(8) \cite{Schwarz:1982jn}, which is the little group of SO(9, 1).
Accordingly, the kinematic structure encoded by $t_8t_8R^4$ is that of the 8d space transverse to a massless momentum.
In the 4-graviton amplitude, the kinematic structure of $t_8t_8 R^4$ is thus determined solely by the 8 transverse components of the graviton polarizations and momenta, rather than all 10.
For example, one is able to extract $t_8t_8 R^4$ from 9d amplitudes obtained by compactifying M-theory on a torus \cite{Green:1997as}.
Consequently, when examining non-vanishing 4-graviton $t_8t_8R^4$ amplitudes, it is ambiguous whether one is investigating the established relation between 11d and 9d, or between 12d and 10d.
By contrast, $\epsilon_{10}\epsilon_{10}R^4$ is intrinsically 10d.
Thus, to strengthen the proposed relation between 12d and 10d amplitudes, it is worth investigating how one may capture $\epsilon_{10}\epsilon_{10}R^4$ from 12d.

Lastly, in 12d $\tau$ should not show up. 
The point in repackaging the type IIB theory and its corrections in a 12d-covariant way is to geometrize $\tau$ as part of the metric, so one should be alarmed to find the need to put in SL(2, Z) covariance by hand, e.g. via $E_{3/2}(\tau,\bar\tau)$.
Perhaps a more appropriate 12d amplitude would be
\begin{equation}
\mathfrak{L}^{(3)} \propto f(\mathcal G_{MN},\mathcal R_{MNPQ}).
\end{equation}
Then upon reduction on a torus, 2 of the 12 directions are singled out, we are thus able to distinguish $\tau$ from the rest of the metric, and obtain the modular function and $R^4$
\begin{equation}
f(\mathcal G_{MN},\mathcal R_{MNPQ})
= E_{3/2}(\tau,\bar\tau)(t_8t_8 + \epsilon_{10}\epsilon_{10})R^4 + ....
\end{equation}
This alternative, more general route may be worth exploring.
One may look into functions that admit expansions over $E_{3/2}$, or consider possible 12d interpretations of the IIB 5-point amplitudes.
The type IIB amplitudes, starting at 5-points, famously contain the ``U(1)-violating terms". This has been identified as a primary obstruction in finding 12d uplifts of effective actions \cite{Liu:2025uqu}.
It would also be illuminating to elucidate how this obstruction shall be interpreted, or worked around, in 12d.






\subsection{KK-modes and D-brane Backgrounds}
Previously we argued that the type IIB effective action at 5-point is critical in validating the 12d repackaging of the 10d effective actions.
The significance of the 5-point amplitudes is further elevated in a separate but closely related context of the type IIB effective actions, namely the role of the supergravity KK-modes as an effective repackaging of the D0 and D(-1) backgrounds in string path integrals. 

We begin in 12d, with coordinates ($x^\mu,y^1,y^2)$, after identifying
\begin{equation}
y^1\to y^1+2\pi R,\quad
(y^1,y^2)\to (y^1+2\pi R\tau_1,y^2+2\pi R\tau_2)
\end{equation}
for some radius $R$, we can Fourier expand a scalar in 12d on $T_2$:
\begin{equation}
\Phi(x^\mu,y^1,y^2)
=\sum_{m,n}\Phi_{m,n}(x^\mu) \times\exp
\left[
\frac{i}{R\tau_2}\left(m\tau_2y^1 + (n-m\tau_1)y^2\right)
\right].
\end{equation}
The massive modes are
\begin{equation}
[-\nabla^2_{10}-\nabla^2_2]\phi_{p,m,n}
=\left(p_{10}^2+M_{m,n}^2\right)\phi_{p,m,n},\quad
M^2_{m,n} 
= \frac{m^2}{R^2}+\frac{(n-m\tau_1)^2}{R^2\tau_2^2}
=\frac{|n-m\tau|^2}{R^2\tau_2^2}.
\end{equation}
For 4-point amplitudes in 10d, we may evaluate contributions from loops of the infinite tower of massive KK-modes using Schwinger proper time
\cite{Green:1997as,Beccaria:2023hhi}
\begin{equation}
\begin{aligned}
\mathcal{A}_4
\propto \sum_{n,m}\int_{\Lambda^{-2}}^\infty
\frac{d\lambda}{\lambda^{3/2}}
e^{-\lambda \frac{|n+m\tau|^2}{R^2\tau_2^2}}
P(s,t;\lambda),\quad
P(s,t;\lambda)
=\int_0^1d\rho_3 \int_0^{\rho_3}d\rho_2\int_0^{\rho_2} d\rho_1e^{-\tau M(s,t;\rho)},\\
M(s,t;\rho)=s\rho_1\rho_2+t\rho_2\rho_3+u\rho_1\rho_3 + t(\rho_1-\rho_2),\quad
s+t+u=0.
\end{aligned}
\label{4-point Schwinger parameter}
\end{equation}
To evaluate $\mathcal{A}_4$ above, one performs Poisson resummations followed by zeta-function renormalization, and a low-energy expansion over $s,t,u$.
But there is a shortcut of adding a spectator dimension to known results of 11d amplitudes on a torus \cite{Green:1997as}.
Either way, one finds
\begin{equation}
\begin{aligned}
\mathcal{A}_4
&\propto  E_{3/2}(\tau,\bar\tau)(s^2+t^2+u^2)+...\\
&\sim E_{3/2}(\tau,\bar\tau)t_8t_8R^4+....
\end{aligned}
\end{equation}
Viewed from string theory, this is the effective action at genus-1 with the D(-1) background.
In other words, the tower of KK-modes on a torus acts as a surrogate for the D(-1) background.
This is in parallel with the 10d type IIA supergravity, where the KK modes on $S_1$ running in a loop produce the D0 background in the type IIA strings \cite{Green:1997as}.

We do not wish to overclaim.
The point we are making is that what used to be a relation between 11d and 9d \cite{Green:1997as} is perfectly compatible with that between 12d and 10d.
But this might just arise from $t_8t_8 R^4$ not being an intrinsically 10d term.
It is worth exploring amplitudes that clearly signal 10d momenta, e.g. $\epsilon_{10}\epsilon_{10}R^4$ at 5-point.
One may attempt to evaluate KK-loop contributions to 5-point amplitudes, using the 5-point Schwinger proper time formula analogous to \eqref{4-point Schwinger parameter}, as given in \cite{Minasian:2015bxa,Green:1999by,Liu:2022bfg}.